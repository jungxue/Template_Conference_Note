\PassOptionsToPackage{unicode=true}{hyperref} % options for packages loaded elsewhere
\PassOptionsToPackage{hyphens}{url}
%
\documentclass[
]{book}
\usepackage{lmodern}
\usepackage{amssymb,amsmath}
\usepackage{ifxetex,ifluatex}
\ifnum 0\ifxetex 1\fi\ifluatex 1\fi=0 % if pdftex
  \usepackage[T1]{fontenc}
  \usepackage[utf8]{inputenc}
  \usepackage{textcomp} % provides euro and other symbols
\else % if luatex or xelatex
  \usepackage{unicode-math}
  \defaultfontfeatures{Scale=MatchLowercase}
  \defaultfontfeatures[\rmfamily]{Ligatures=TeX,Scale=1}
\fi
% use upquote if available, for straight quotes in verbatim environments
\IfFileExists{upquote.sty}{\usepackage{upquote}}{}
\IfFileExists{microtype.sty}{% use microtype if available
  \usepackage[]{microtype}
  \UseMicrotypeSet[protrusion]{basicmath} % disable protrusion for tt fonts
}{}
\makeatletter
\@ifundefined{KOMAClassName}{% if non-KOMA class
  \IfFileExists{parskip.sty}{%
    \usepackage{parskip}
  }{% else
    \setlength{\parindent}{0pt}
    \setlength{\parskip}{6pt plus 2pt minus 1pt}}
}{% if KOMA class
  \KOMAoptions{parskip=half}}
\makeatother
\usepackage{xcolor}
\IfFileExists{xurl.sty}{\usepackage{xurl}}{} % add URL line breaks if available
\IfFileExists{bookmark.sty}{\usepackage{bookmark}}{\usepackage{hyperref}}
\hypersetup{
  pdftitle={Study note template},
  pdfauthor={Jung Xue},
  pdfborder={0 0 0},
  breaklinks=true}
\urlstyle{same}  % don't use monospace font for urls
\usepackage{longtable,booktabs}
% Allow footnotes in longtable head/foot
\IfFileExists{footnotehyper.sty}{\usepackage{footnotehyper}}{\usepackage{footnote}}
\makesavenoteenv{longtable}
\usepackage{graphicx,grffile}
\makeatletter
\def\maxwidth{\ifdim\Gin@nat@width>\linewidth\linewidth\else\Gin@nat@width\fi}
\def\maxheight{\ifdim\Gin@nat@height>\textheight\textheight\else\Gin@nat@height\fi}
\makeatother
% Scale images if necessary, so that they will not overflow the page
% margins by default, and it is still possible to overwrite the defaults
% using explicit options in \includegraphics[width, height, ...]{}
\setkeys{Gin}{width=\maxwidth,height=\maxheight,keepaspectratio}
\setlength{\emergencystretch}{3em}  % prevent overfull lines
\providecommand{\tightlist}{%
  \setlength{\itemsep}{0pt}\setlength{\parskip}{0pt}}
\setcounter{secnumdepth}{5}
% Redefines (sub)paragraphs to behave more like sections
\ifx\paragraph\undefined\else
  \let\oldparagraph\paragraph
  \renewcommand{\paragraph}[1]{\oldparagraph{#1}\mbox{}}
\fi
\ifx\subparagraph\undefined\else
  \let\oldsubparagraph\subparagraph
  \renewcommand{\subparagraph}[1]{\oldsubparagraph{#1}\mbox{}}
\fi

% set default figure placement to htbp
\makeatletter
\def\fps@figure{htbp}
\makeatother

\usepackage{booktabs}
\usepackage[]{natbib}
\bibliographystyle{apalike}

\title{Study note template}
\author{Jung Xue}
\date{2020-11-23}

\begin{document}
\maketitle

{
\setcounter{tocdepth}{1}
\tableofcontents
}
\hypertarget{conference-information}{%
\chapter*{Conference information}\label{conference-information}}
\addcontentsline{toc}{chapter}{Conference information}

\textbf{XXXX Conference:}

\begin{itemize}
\tightlist
\item
  \textbf{Time:} xx:00pm/xx/xx/xx
\item
  \textbf{Venue:} The university of Auckland 303S-310 \href{}{Map}
\item
  \textbf{Registartion:} xx/xx/xx\\
\item
  \textbf{Hosted by:} \href{}{Host association website}
\end{itemize}

\textbf{Keynote Speakers:}

\begin{longtable}[]{@{}llll@{}}
\toprule
Speaker & topic & Email & Website\tabularnewline
\midrule
\endhead
Joe Smith & Pickles & \href{mailto:xx@gmail.com}{\nolinkurl{xx@gmail.com}} & \href{www.google.com}{Uni}\tabularnewline
& & &\tabularnewline
& & &\tabularnewline
\bottomrule
\end{longtable}

\textbf{interesting I have meet/noticed}

\begin{longtable}[]{@{}llll@{}}
\toprule
People & Field/Job & Contact & Facts\tabularnewline
\midrule
\endhead
Joe Smith & Consultant @ UoA & \href{mailto:xx@gmail.com}{\nolinkurl{xx@gmail.com}} & His from New Caledonia\tabularnewline
& & &\tabularnewline
& & &\tabularnewline
\bottomrule
\end{longtable}

Note: All information disclosed within this conference e-note are intented for personal use.

\hypertarget{speaker-1}{%
\chapter*{Speaker 1}\label{speaker-1}}
\addcontentsline{toc}{chapter}{Speaker 1}

\hypertarget{subsection}{%
\section{Subsection}\label{subsection}}

\hypertarget{speaker-2}{%
\chapter*{Speaker 2}\label{speaker-2}}
\addcontentsline{toc}{chapter}{Speaker 2}

\hypertarget{subsection}{%
\section{Subsection}\label{subsection}}

\hypertarget{speaker-3}{%
\chapter*{Speaker 3}\label{speaker-3}}
\addcontentsline{toc}{chapter}{Speaker 3}

\hypertarget{subsection}{%
\section{Subsection}\label{subsection}}

\hypertarget{speaker-4}{%
\chapter*{Speaker 4}\label{speaker-4}}
\addcontentsline{toc}{chapter}{Speaker 4}

\hypertarget{subsection}{%
\section{Subsection}\label{subsection}}

\hypertarget{speaker-5}{%
\chapter*{Speaker 5}\label{speaker-5}}
\addcontentsline{toc}{chapter}{Speaker 5}

\hypertarget{subsection}{%
\section{Subsection}\label{subsection}}

\#Speaker 6 \{-\}

\hypertarget{subsection}{%
\section{Subsection}\label{subsection}}

\hypertarget{speaker-7}{%
\chapter*{Speaker 7}\label{speaker-7}}
\addcontentsline{toc}{chapter}{Speaker 7}

\hypertarget{subsection}{%
\section{Subsection}\label{subsection}}

\hypertarget{speaker-8}{%
\chapter*{Speaker 8}\label{speaker-8}}
\addcontentsline{toc}{chapter}{Speaker 8}

\hypertarget{subsection}{%
\section{Subsection}\label{subsection}}

\hypertarget{speaker-9}{%
\chapter*{Speaker 9}\label{speaker-9}}
\addcontentsline{toc}{chapter}{Speaker 9}

\hypertarget{subsection}{%
\section{Subsection}\label{subsection}}

\hypertarget{speaker-10}{%
\chapter*{Speaker 10}\label{speaker-10}}
\addcontentsline{toc}{chapter}{Speaker 10}

\hypertarget{subsection}{%
\section{Subsection}\label{subsection}}

\hypertarget{speaker-11}{%
\chapter*{Speaker 11}\label{speaker-11}}
\addcontentsline{toc}{chapter}{Speaker 11}

\hypertarget{subsection}{%
\section{Subsection}\label{subsection}}

\hypertarget{speaker-12}{%
\chapter*{Speaker 12}\label{speaker-12}}
\addcontentsline{toc}{chapter}{Speaker 12}

\hypertarget{subsection}{%
\section{Subsection}\label{subsection}}

\hypertarget{speaker-13}{%
\chapter*{Speaker 13}\label{speaker-13}}
\addcontentsline{toc}{chapter}{Speaker 13}

\hypertarget{subsection}{%
\section{Subsection}\label{subsection}}

\hypertarget{speaker-14}{%
\chapter*{Speaker 14}\label{speaker-14}}
\addcontentsline{toc}{chapter}{Speaker 14}

\hypertarget{subsection}{%
\section{Subsection}\label{subsection}}

\hypertarget{speaker-15}{%
\chapter*{Speaker 15}\label{speaker-15}}
\addcontentsline{toc}{chapter}{Speaker 15}

\hypertarget{subsection}{%
\section{Subsection}\label{subsection}}

\hypertarget{subsection}{%
\section{Subsection}\label{subsection}}

\hypertarget{speaker-17}{%
\chapter*{Speaker 17}\label{speaker-17}}
\addcontentsline{toc}{chapter}{Speaker 17}

\hypertarget{subsection}{%
\section{Subsection}\label{subsection}}

\hypertarget{speaker-18}{%
\chapter*{Speaker 18}\label{speaker-18}}
\addcontentsline{toc}{chapter}{Speaker 18}

\hypertarget{subsection}{%
\section{Subsection}\label{subsection}}

\hypertarget{speaker-19}{%
\chapter*{Speaker 19}\label{speaker-19}}
\addcontentsline{toc}{chapter}{Speaker 19}

\hypertarget{subsection}{%
\section{Subsection}\label{subsection}}

\hypertarget{speaker-20}{%
\chapter*{Speaker 20}\label{speaker-20}}
\addcontentsline{toc}{chapter}{Speaker 20}

\hypertarget{subsection}{%
\section{Subsection}\label{subsection}}

\hypertarget{concluding-remarks}{%
\chapter*{Concluding Remarks}\label{concluding-remarks}}
\addcontentsline{toc}{chapter}{Concluding Remarks}

What did you learnt by the end of this session/course?

Take home message?

Add 3 questions to ponder.

\hypertarget{how-to-use-rbookdown}{%
\chapter*{How to use RBookDown}\label{how-to-use-rbookdown}}
\addcontentsline{toc}{chapter}{How to use RBookDown}

Firstly, you will have to read the \href{https://bookdown.org/yihui/bookdown/}{RBookDown Bible} by YiHui Xie

In essence, you write in a mixture of markdown (For basics), html (to extend on markdown) and latex language (mostly for equations) to create a simple Note.

You can customise your style and theme through your own CSS.

RMarkdown are mostly used to knit e-books(HTML), use TexStudio if you want a proper PDF, it is easier.

\textbf{Here are some useful tips to get started}

1: To add a chapter, just open a R file and save as \texttt{.RMD}. Use number 0 to 99 with a hyphen \texttt{-} to order the RMD files and maybe add a Chapter name so it is easier to select from \texttt{Files} window at bottom right of the R Studio.

2: Code chunks can generate graphical outputs, To insert pictures just use \texttt{include\_graphics} instead of \texttt{\textbackslash{}includegraphics\{\}} or \texttt{!{[}{]}()}. Width can be customised.

\begin{verbatim}
knitr::include_graphics(rep('images/knit-logo.png', 3))
\end{verbatim}

3: Use 1 grave accent ` to include the inline code, use 3 grave accent to include a chunk of code.

  \bibliography{references.bib}

\end{document}
